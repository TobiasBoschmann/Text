\documentclass{article}

\usepackage{amsmath}
\usepackage[version=4]{mhchem}
\usepackage{longtable}	% For tables longer than a page
\usepackage{multirow}	% Allows multirow in tables
\usepackage{multicol}	% Allows multicolumn in tables
\usepackage{booktabs}	% Better horizontal lines in tables \toprule \midrule ...

\usepackage{dcolumn} 	% Allows allignment by decimal point
  \newcolumntype{d}[1]{D{.}{.}{#1}}	% Defining some new formats
  \newcolumntype{C}[1]{>{\centering\let\newline\\\arraybackslash\hspace{0pt}}m{#1}}



\begin{document}

\begin{table}[ht]
		\centering
		\caption{Standardtbildungsenthalpien und Standardtentropien für  [\cite{?%Entwicklung und Charakterisierung von Elektroden für die Sauerstoffentwicklung in der alkalischen Wasserelektrolyse
		}] sowie stöchometrische Koeffizienten aus (3.1)}
		\begin{tabular}{c c c c}
		\toprule
		\multirow{2}{*}{Komponenten i} & 
		\multicolumn{1}{c}{$\Delta_f H^0_{f,i}$} & 
		\multicolumn{1}{c}{$S^0_i$} &
		\multicolumn{1}{c}{$\nu_i$}
		\\
		& 
		\multicolumn{1}{c}{$\textrm{[kJ/mol]}$}& 
		\multicolumn{1}{c}{$\textrm{[J/(molK)]}$} &
		\multicolumn{1}{c}{$\textrm{[--]}$}
		\\
		\midrule
		$\ce{H2O}$ & -285,25 & 70,12 & -1\\
		$\ce{O2}$ & 0 & 205,25 & 2\\
		$\ce{H2}$ & 0 & 130,7 & 1\\
		\bottomrule
		\end{tabular}
		\label{tab:rule}
		\end{table}		

\end{document}