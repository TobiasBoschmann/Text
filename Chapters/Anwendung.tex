\chapter{Anwendung}
\label{cha:Anwendung}

\section{Simulation der Wasserstoffkomponenten}

\subsection{Parameter des Alkalischen Elektrolyseurs}
-Gewählte Parameter und Annahmen für die Simulation aufführen und begründen\\
(Membrandicke, Lambda usw., Temperaturführung;)\\
Tmin=20C; ->50-60C normal; Wahl von Tmax!\\
A*Zellanzahl aus n-max und i-max ableiten\\

Alkalischer Elektrolyseur:\\
-20 \% NaOH.\\
-n-max = 21,33 Nmc/h(Normkubikmeter/Stunde) = 0.2643 mol/s \\
-A aus i-max und n abgeschätzt ca.10 m2\\



-Validierung, Membrandicke: 100-200 mü m \cite{rashid - Hydrogen Production by Water Electrolysis: A
Review of Alkaline Water Electrolysis, PEM Water
Electrolysis and High Temperature Water
Electrolysis}\\


Daten aus Elektolyse-Modelica (PEM):\\
$i_0 = \SI{1,08e-4}{\A\per\m\squared}$

$\alpha = 0,7353$\\

Daten aus Modelling and simulation of an alkaline electrolyser cell (alk):\\
$i_0 = \SI{1e-10}{\A\per\cm\squared}$

$\alpha = 1.75$\\

nach Milewski et al.:\\
$delta-alk = 0,66cm$\\
andere ohmsche Verluste um Faktor 300 kleiner -> vernachlässigen\\ 

Thermische Kapazität in Elektrolyseur Modelica\\
-> Man könnte Cp über die Leistung skalieren\\ 
Cp nach Elektrolyseur-Modelica 162116 J/K bei PEM mit A=60*290e-4\\

Q-verlust aus Elektrolyseur-Modelica: $1/R_(th)(T-T_u)$ (mit $R_(th) = 0,0668 K/W$ und $C_p = 162116 J/K$ bei $60 \cdot 290 cm^2$)\\


\subsection{Parameter der Brennstoffzelle}
-Gewählte Parameter und Annahmen für die Simulation aufführen und begründen\\
(Membrandicke, Lambda usw., Temperaturführung;)\\

\section{Entwicklung von Energiesystem-Konzepten}
-Systemrandbedingungen festlegen\\
(H-Bedarf, Input-Temperaturen, Wärmebedarf, Stromkosten...)\\
-Festlegen von Parametern für Elektrolyseur, Brennstoffzelle usw. für die Simulation (Welche Leistungsklasse etc.)\\
30\% der Dachfläche (1500m*m) -> 450m*m\\
