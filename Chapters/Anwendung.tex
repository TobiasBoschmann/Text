\chapter{Anwendung}
\label{cha:Anwendung}

\section{Modelle der Wasserstoffkomponenten}

\subsection{Modellierung des Alkalischen Elektrolyseurs}
-Gewählte Parameter und Annahmen für die Simulation aufführen und begründen\\
(Membrandicke, Lambda usw., Temperaturführung;)\\
Tmin=20C; ->50-60C normal; Wahl von Tmax!\\
A*Zellanzahl aus n-max und i-max ableiten\\

Alkalischer Elektrolyseur:\\
-20 \% NaOH.\\
-nMaxO2 = 18,2 Liter/Stunde\\
-P-el-ges-max = 108kW\\
-I-max = 165A\\
-dicke: 100-200 mü m \cite{rashid - Hydrogen Production by Water Electrolysis: A
Review of Alkaline Water Electrolysis, PEM Water
Electrolysis and High Temperature Water
Electrolysis}


Thermische Kapazität in Elektrolyseur Modelica\\
-> Man könnte Cp über die Leistung skalieren\\ 



\subsection{Modellierung der Brennstoffzelle}
-Gewählte Parameter und Annahmen für die Simulation aufführen und begründen\\
(Membrandicke, Lambda usw., Temperaturführung;)\\

\section{Entwicklung von Energiesystem-Konzepten}
-Systemrandbedingungen festlegen\\
(H-Bedarf, Input-Temperaturen, Wärmebedarf, Stromkosten...)\\
-Festlegen von Parametern für Elektrolyseur, Brennstoffzelle usw. für die Simulation (Welche Leistungsklasse etc.)\\

