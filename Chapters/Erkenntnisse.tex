\chapter{Fazit und Ausblick}
\label{cha:Fazit}
In der vorliegenden Arbeit wird das Wasserstoff-Energiesystem eines Quarzglasherstellers  betrachtet, um Konzepte zur effizienteren Gestaltung des Systems zu identifizieren:\\

Im ersten Schritt wird ein physikalisches Modell für Elektrolyseure und Brennstoffzellen in Modelica entwickelt. Das Modell wird anhand von Kennwerten aus Datenblättern parametriert und ist darüber hinaus in der Lage, Zellen verschiedener Technologien zu modellieren.\\
Aus der Validierung der Modelle geht hervor, dass sie für die im Rahmen dieser Arbeit angestrebte Untersuchungen ausreichend genaue Ergebnisse liefern: Bei der für die Simulation gewählten Betriebstemperatur liegen die Abweichungen der Polarisationskurve des alkalischen Elektrolyseurs bei unter $\SI{3,5}{\%}$, für die PEM-Brennstoffzelle ergeben sich Abweichungen von durchschnittlich $\SI{5,3}{\%}$.\\
Für einen PEM-Elektrolyseur werden Abweichungen von unter  $\SI{0,5}{\%}$ zu Messwerten aus der Literatur verzeichnet.\\
Bei den beiden PEM-Zellen nimmt die Genauigkeit des Modells unterhalb der gewählten Betriebstemperatur stark ab.\\

Als Erweiterungen des Wasserstoff-Energiesystems des Quarzglasherstellers werden folgende Komponenten betrachtet: Eine Brennstoffzelle zur Nutzung des Wasserstoffüberschusses durch Kraft-Wärme-Kopplung, eine PV-Anlage und ein Gasspeichers zum Einspeichern von Solarstrom-Überschuss. Zur Bewertung der Systemkonzepte werden der Kapitalwert und die \ce{CO2}-Einsparungen berechnet.\\
Die ökonomische Bewertung der Brennstoffzelle fällt tendenziell negativ aus: Abhängig vom prognostizierten Prozessgasbedarf liegt der Kapitalwert zwischen $-15.388$ und $5.\SI{847}{\sieuro}$. Aus ökologischer Sicht stellt die Installation der Brennstoffzelle eine Verbesserung dar - es ergeben sich \ce{CO2}-Einsparungen von $3,6$ bis $\SI{8}{\tonne}$. Die Optimierung des Brennstoffzellen-Konzepts verspricht eine verbesserte ökonomische Bewertung bei nahezu gleichbleibenden \ce{CO2}-Emissionen.\\ 
Die PV-Anlage wird als sinnvoll Investition gewertet, da der Kapitalwert mit $84.\SI{726}{\sieuro}$ klar positiv ausfällt (Der Kapitalwert übersteigt dabei die Anfangsinvestitionen). Auch aus ökologischer Sicht wird die PV-Anlage verglichen mit der Brennstoffzelle positiver eingestuft: Die simulierten \ce{CO2}-Einsparungen sind mit $\SI{24,8}{\tonne}$ um den Faktor 3 bis 6 höher.\\

Im Bezug auf die Modellierung der Wasserstoffkomponenten ergeben sich aus dieser Arbeit für die Zukunft folgende Arbeitsschritte:
\begin{itemize}
\item Um dynamische Vorgänge in dem Modell zu implementieren, kann die Instaionarität der Zelltemperatur in der Energiebilanz (Gleichung \ref{gl:Energiebilanz}) berücksichtigt werden. Darüber hinaus ist es sinnvoll, die Modellierung des thermischen Verhaltens der Brennstoffzelle auch in das Elektrolyseur-Modell zu implementieren.
\item Ein Modellierung des Temperatureinflusses auf die Durchtrittsfaktor oder die Austauschstromdichte verspricht eine höhere Genauigkeit des Modells für PEM-Zellen bei niedrigen Temperaturen. Weiterhin bleibt zu untersuchen, ob für Elektrolyseure und Brennstoffzellen gleicher Bauart die Werte der Fitting-Parameter unverändert bleiben. Gegebenenfalls sollte das Modell um eine getrennte Parametrierung von Brennstoffzellen und Elektrolyseuren erweitert werden.
\item Die Modellierung von Konzentrationsüberspannungen ist zur Verminderung der Abweichungen des Brennstoffzellen-Modells im oberen Lastbereich eine sinnvolle Erweiterung.
\item Für folgende Arbeiten wird eine Validierung des Modells bei verschiedenen Druckniveaus empfohlen.\\
\end{itemize}
Im Hinblick auf die Identifikation sinnvoller Konzepte des Wasserstoff-Energiesystems des Quarzglasherstellers sind in Zukunft folgende Schritte von Bedeutung:
\begin{itemize}
\item Aufbauend auf den Simulationsergebnissen dieser Arbeit sollte eine genauere Evaluierung der als sinnvoll bewerteten Systemkonzepte erfolgen. Insbesondere Simulationen mit realen Messreihen des Prozessgasbedarfs steigern dabei die Aussagekraft der Ergebnisse.
\item Eine Optimierung des Brennstoffzellen-Konzepts durch die Erweiterung um einen Wasserstoffspeicher verspricht aus ökonomischer Sicht deutliche Verbesserungen. Daher ist in folgenden Arbeiten eine optimierte Auslegung des Brennstoffzellen-Konzepts sowie eine Bewertung desselben auf der Grundlage einer Simulationsstudie durchzuführen.
\end{itemize}