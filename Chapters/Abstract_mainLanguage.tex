Bei der Quarzglasveredelung wird in fluktuierenden Mengen Wasserstoff sowie Sauerstoff benötigt. Diese werden von dem Kunden intern mithilfe eines Elektrolyseurs erzeugt. Elektrolyseure produzieren Wasserstoff und Sauerstoff in einem festen Stoffmengenverhaltnis, allerdings wird in einigen Verfahrensschritten des Kunden ein anderes Stoffmengenverhältnis benötigt. Daher werden für den Betrieb nicht benötigte Mengen von Wasserstoffs erzeugt, welche im aktuellen System ungenutzt in die Atmosphäre entweichen. In dieser Arbeit wird anhand einer Simulationsstudie untersucht, ob eine Nutzung der überschüssigen Gase sinnvoll in das Energiesystem des Herstellers eingebunden werden kann. Dabei wird insbesondere die Verwendung einer Brennstoffzelle zur Kraft-Wärme-Kopplung betrachtet. Zudem wird untersucht, welchen Nutzten die Einbindung einer Photovoltaik-Anlage in das Energiesystem für die Deckung des Energiebedarfs hat.\\

In vorangehenden Arbeiten wurde an dem Institut bereits die Modelica-Bibliothek AixLib entwickelt, welche unter anderem Modelle der relevanten Elektronik-Komponenten, wie Beispielsweise der PV-Anlage, enthält. Im ersten Schritt werden für die benötigten Wasserstoffkomponenten - insbesondere Elektrolyseur und Brennstoffzelle- Modelle in Modelica erstellt. Dabei ist ein Aufbau erstrebenswert, der es ermöglicht, verschiedene Anlagen anhand von Kennwerten aus Datenblättern zu modellieren. Daraufhin werden mögliche Szenarien des Energiesystems sowie relevante Bewertungsgrößen herausgearbeitet. Im folgenden Schritt werden die Szenarien in Modelica simuliert und abschließend wird eine Interpretation der Simulationsergebnisse und darauf aufbauend eine Bewertung der Szenarien anhand der gewählten Kriterien durchgeführt.\\
Ergebnis der Arbeit sind somit einerseits Modelle der Wasserstoffkomponenten, welche auf der bestehenden AixLib-Bibliothek aufbauen und für folgende Arbeiten genutzt werden können. Andererseits wird eine auf Simulationsergebnissen gestützte Bewertung verschiedener Wasserstoff-Energie Systeme anhand von definierten Kriterien durchgeführt.