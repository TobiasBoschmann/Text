Bei der Quarzglasveredelung wird in fluktuierenden Mengen Wasserstoff sowie Sauerstoff benötigt. Diese werden im Wasserstoff-Energiesystem eines Quarzglasherstellers mithilfe eines Elektrolyseurs erzeugt. Elektrolyseure produzieren Wasserstoff und Sauerstoff in einem festen Stoffmengenverhältnis, allerdings wird in einigen Verfahrensschritten ein anderes Stoffmengenverhältnis benötigt. Daher werden für den Betrieb nicht benötigte Mengen Wasserstoff erzeugt, welche im aktuellen System ungenutzt in die Atmosphäre entweichen. In dieser Arbeit wird daher anhand einer Simulationsstudie untersucht, ob eine Nutzung der überschüssigen Gase sinnvoll in das Energiesystem des Herstellers eingebunden werden kann. Als Erweiterung des aktuellen Wasserstoff-Energiesystems wird eine Brennstoffzelle, eine Photovoltaik-Anlage und ein Zwischenspeicher für die Prozessgase in Betracht gezogen.\\

In vorangehenden Arbeiten wurde bereits die Modelica-Bibliothek AixLib entwickelt, welche unter anderem Modelle der relevanten Elektronik-Komponenten, wie Beispielsweise der Photovoltaik-Anlage, enthält. Im ersten Schritt werden für die benötigten Wasserstoffkomponenten - insbesondere Elektrolyseur und Brennstoffzelle - Modelle in Modelica erstellt. Dabei wird ein Aufbau gewählt, der es ermöglicht, verschiedene Anlagen anhand von Kennwerten aus Datenblättern zu modellieren. 
Aus einer Validierung der Komponentenmodelle geht hervor, dass diese für die in dieser Arbeit angestrebten Untersuchungen geeignet sind: In den gewählten Betriebspunkten ergeben sich für den Elektrolyseur und die Brennstoffzelle Abweichungen von durchschnittlich $\SI{2,0}{\%}$ und $\SI{5,3}{\%}$ der relevanten Ausgangsgrößen zu Messwerten aus der Literatur.\\

Im zweiten Schritt werden mögliche Konzepte des Energiesystems herausgearbeitet. Dabei wird zur Nutzung des Wasserstoffüberschusses die Verwendung einer Brennstoffzelle betrachtet. Zudem wird die Einbindung einer Photovoltaik-Anlage zur Deckung des Strombedarfs sowie deren Ergänzung um einen Gasspeicher untersucht. Daraufhin werden die Konzepte in Modelica simuliert und abschließend wird eine Interpretation der Simulationsergebnisse durchgeführt.
Zur Bewertung werden die Kapitalwerte sowie die $\ce{CO2}$-Einsparungen der Systemkonzepte verglichen.\\

Die Investition in eine Brennstoffzelle stellt sich als tendenziell unwirtschaftlich heraus, wobei die ökologische Bewertung positiv ausfällt. Eine Optimierung des Brennstoffzellen-Konzepts verspricht deutliche Verbesserungen des Kapitalwerts, ohne die $\ce{CO2}$-Einsparungen signifikant zu vermindern. Die PV-Anlage stellt sowohl nach der ökonomischer, als auch nach der ökologischer Bewertung die sinnvollste Erweiterung des aktuellen Systems dar. Die Ergänzung der PV-Anlage um einen Gasspeicher hat auf die Ausgangsgrößen der Simulation keinen Einfluss und wird daher als nicht sinnvoll bewertet.