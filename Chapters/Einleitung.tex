\chapter{Einleitung}
\label{cha:Einleitung}
Die deutsche Bundesregierung hat durch den Beschluss vom 12.05.2021 die Zielvorgaben der $\ce{CO2}$ Einsparungen weiter verschärft. Waren anfangs $\SI{55}{\%}$ Einsparungen im Jahr 2030 im Vergleich zum $\ce{CO2}$-Ausstoß von 1990 geplant, wurde dieser Wert auf $\SI{65}{\%}$ angehoben \citep{bundesregierung_deutschland_klimaschutzgesetz_nodate}. Auch die Industrie ist gefordert, Maßnahmen zur $\ce{CO2}$-Einsparung zu entwickeln und umzusetzen: In den nächsten 10 Jahren sollten die jährlichen Emissionen der deutschen Industriebranche nach dem alten Plan um $\SI{21}{\%}$ gesenkt werden \citep{energie_treibhausgasemissionen_nodate}, was sich durch den aktuellen Beschluss weiter verschärft. In der vorliegenden Arbeit wird daher exemplarisch das Wasserstoff-Energiesystem eines Quarzglasherstellers betrachtet, um Konzepte zur effizienteren Gestaltung solcher Systeme zu identifizieren.\\

Bei der Nachbehandlung von Quarzglas werden Wasserstoff sowie Sauerstoff benötigt welche im Wasserstoff-Energiesystem des Herstellers mithilfe eines Elektrolyseurs erzeugt werden. Elektrolyseure spalten Wasser zu Wasserstoff und Sauerstoff und produzieren die beiden Gase daher in einem festen Stoffmengenverhältnis, allerdings werden die Gase in den Verfahrensschritten in einem anderen Stoffmengenverhältnis benötigt. Somit wird ein Wasserstoffüberschuss erzeugt, welcher ungenutzt in die Atmosphäre entweicht.\\

In dieser Arbeit wird daher anhand einer Simulationsstudie untersucht, ob das Wasserstoff-Energiesystem des Herstellers effizienter gestaltet werden kann. Dabei wird die Verwendung einer Brennstoffzelle zur Nutzung der überschüssigen Gase betrachtet. Diese bietet die Möglichkeit, den Wasserstoff  mit Umgebungsluft unter Freisetzung von Wärme und elektrischem Strom zu Wasser zu kombinieren.
Zudem wird untersucht, welchen Nutzten die Einbindung einer Photovoltaik-Anlage in das Energiesystem für die Deckung des Strombedarfs hat.\\

Zur Modellierung der Elektronik-Komponenten, wird auf vorangehende Arbeiten zurückgegriffen. Am Lehrstuhl für Gebäude- und Raumklimatechnik wurde bereits eine Bibliothek in der Modellierungsprache \textit{Modelica} entwickelt, welche unter anderem Modelle für Photovoltaik(PV)-Anlagen enthält. Da die Bibliothek keine Modelle der Wasserstoffkomponenten beinhaltet, ist die Modellierung von Elektrolyseuren und Brennstoffzellen Bestandteil dieser Arbeit. Für eine bessere Wiederverwendbarkeit und vereinfachte Nutzung des Modells wird ein Aufbau gewählt, der es ermöglicht, verschiedene Anlagen anhand von Kennwerten aus Datenblättern zu modellieren. Die Eignung der entwickelten Modelle für die folgenden Simulationen wird durch eine Validierung mit Literaturangaben sichergestellt. Auf die Simulation der betrachteten Wasserstoff-Energiesysteme folgt die Interpretation der Simulationsergebnisse. Um ein Bewertung inklusive Vergleich der Systemkonzepte zu ermöglichen, werden Bewertungsgrößen festgelegt, die aus den Simulationsergebnissen berechnet werden.\\

Ergebnis der Arbeit sind somit im ersten Schritt Modelle der Wasserstoffkomponenten, welche auf der bestehenden \textit{Modelica}-Bibliothek basieren und für folgende Arbeiten genutzt werden können. Darauf aufbauend wird eine auf Simulationsergebnissen gestützte Bewertung verschiedener Wasserstoff-Energie Systeme anhand von definierten Bewertungsgrößen durchgeführt. Dazu ist die vorliegende Arbeit wie folgt aufgebaut:

\begin{itemize}
\item In Kapitel 2 werden die allgemeinen Grundlagen vorgestellt, auf denen die Modellierung der Wasserstoffkomponenten aufbaut. Anfangs werden dazu Grundlagen der Modellierung sowie der Stand der Technik Modellierung von Elektrolyseuren und Brennstoffzellen vorgestellt. Weiterhin wird die Funktionsweise von Elektrolyse- und Brennstoffzellen, deren thermodynamische und elektrochemische Grundlagen sowie gängige technische Lösungen erläutert.
\item Die Modellierung der Wasserstoff-Energiesysteme wird Kapitel 3 beschrieben. Anfangs werden die Modelle des Elektrolyseurs und der Brennstoffzelle vorgestellt. Weiterhin werden Bewertungskriterien für Wasserstoff-Energiesysteme ausgearbeitet und anschließend alternative Systemkonzepte präsentiert. Im letzten Abschnitt werden die für die Simulation der Systemkonzepte gewählten Parameter und Randbedingungen hergeleitet.
\item In Kapitel 4 wird einerseits eine Validierung des Elektrolyseur- und Brennstoffzellenmodells anhand von in der Literatur dokumentierten Messwerten vorgenommen. Weiterhin werden die Simulationsergebnisse präsentiert und darauf aufbauend eine Bewertung der Systemkonzepte nach den herausgearbeiteten Kriterien vorgenommen.
\end{itemize}