\chapter*{Nomenklatur}
\begin{onehalfspacing}
\begin{longtable}[h]{p{0.15\textwidth} p{0.65\textwidth} p{0.1\textwidth}}
		\caption*{\textbf{Formelzeichen und Einheiten}} \\
		\\
		\textbf{Symbol} & \textbf{Bedeutung} & \textbf{Einheit} \\ %\hline 
		\endhead
		\\
		\multicolumn{3}{c}{Fortsetzung auf der nächsten Seite} \\
		\endfoot
		\multicolumn{3}{c}{ } \\
		\endlastfoot
		
		$A$ 		& Fläche 											& \si{\m^{2}}\\
		$c_{p}$		& spezifische Wärmekapazität bei konstantem Druck	& \si{\joule\per(\kilogram\kelvin)}\\
		$C$			& Wärmekapazität									& \si{\watt\per\kilogram}\\
		$H $ 		& Enthalpie 										& \si{\joule}\\		
		$\dot{H}$ 	& Enthalpiestrom 									& \si{\joule\per\second}\\
		$E$ 		& Exergie 											& \si{\joule}\\
		$e$ 		& spezifische Exergie 								& \si{\joule\per\kilogram}\\
		$\dot{m}$ 	& Massenstrom 										& \si{\kilogram\per\second}\\
		$p$ 		& Druck 											& \si{\pascal}\\
		$\dot{Q}$ 	& Wärmestrom 										& \si{\watt}\\
		$R$ 		& spezifische Gaskonstante 							& \si{\joule\per(\kilogram\kelvin)}\\
		$S$ 		& Entropie 											& \si{\joule\per\kelvin}\\
		$\dot{S}$ 	& Entropiestrom 									& \si{\watt\per\kelvin}\\
		$T$ 		& Temperatur 										& \si{\kelvin}\\
		$t$ 		& Zeit 												& \si{\second}\\
		$U$ 		& innere Energie 									& \si{\joule}\\
		$U_{T}$ 	& Wärmedurchgangskoeffizient 						& \si{\watt\per(\kilogram\kelvin)}\\
		$h$ 		& Wärmeübergangskoeffizient 						& \si{\watt\per(\m^{2}\kelvin)}\\		
		$V$ 		& Volumen 											& \si{\meter^{3}}\\
		$\dot{V}$	& Volumenstrom										& \si{\m^3\per\second}\\
		$\dot{W}$ 	& Leistung 											& \si{\watt}\\
		$Y$ 		& Wasserbeladung der Luft 							& \si{\gram\per\kilogram}\\
		
		
\end{longtable}

\begin{longtable}[h]{p{0.15\textwidth} p{0.65\textwidth} p{0.1\textwidth}}
		\caption*{\textbf{Griechische Formelzeichen}} \\
		\\
		\textbf{Symbol} & \textbf{Bedeutung} & \textbf{Einheit} \\ %\hline 
		\endhead
		\\
		\multicolumn{3}{c}{Fortsetzung auf der nächsten Seite} \\
		\endfoot
		\multicolumn{3}{c}{ } \\
		\endlastfoot
		
		$\eta_{C}$ 				& Carnot-Wirkungsgrad 							& ---\\
		$\kappa_{\mathrm{E}}$ 	& exergetische Aufwandszahl der Wärmeerzeugung 	& ---\\
		$\kappa_{\mathrm{T}}$ 	& exergetische Aufwandszahl des Wärmetransfers 	& ---\\
		$\Phi$ 					& thermische Leistung 							& \si{\watt}\\
		$\varrho$				& Massendichte									& \si{\kilogram\per \metre^3}\\
		$\sigma$				& Temperaturspreizung							& \si{\kelvin}\\
		$\vartheta $ 			& Temperatur  									& \si{\celsius}\\
		$\Delta\vartheta $ 		& Temperaturdifferenz  							& \si{\kelvin}\\
		
\end{longtable}

\begin{longtable}[h]{p{0.15\textwidth} p{0.75\textwidth}}
		\caption*{\textbf{Indizes und Abkürzungen}} \\
		\\
		\textbf{Symbol} & \textbf{Bedeutung} \\ %\hline 
		\endhead
		\\
		\multicolumn{2}{c}{Fortsetzung auf der nächsten Seite} \\
		\endfoot
		\multicolumn{2}{c}{ } \\
		\endlastfoot
		
		0 		& Referenzzustand (\emph{ambient dead state})\\
		A 		& Außen/Umgebung\\ 		
		CH 		& chemisch\\
		CV 		& Kontrollvolumen (\emph{control volume})\\
		DSC 	& Dynamische Differenzkalorimetrie (\emph{differential scanning calorimetry}) \\
		e 		& über die Systemgrenze (\emph{external})\\
		F 		& Volumenstrom\\	
		FW 		& Fassadenwärmeübertrager\\
		gen 	& erzeugt (\emph{generated})\\
		In 		& Eingang (\emph{input})\\
		KN 		& kinetisch\\
		KRM 	& Kapillarrohrmatte\\
		LabVIEW & Programmiersprache und Entwicklungsumgebung für die Messdatenerfassung der Firma National Instruments\\
		L 		& Luft\\
		LWS 	& Latentwärmespeicher\\	
		m 		& Mittelwert\\
		Ob		&Oberfläche\\
		PCM 	& Latentwärmespeichermaterial (\emph{phase change material})\\
		PH 		& physikalisch\\
		PT 		& potentiell\\
		Q 		& auf einen Wärmestrom bezogen\\
		R 		& Rücklauf\\
		Reg		& Speicherregeneration\\
		T 		& Temperatur\\
		$\Delta$ t & Zeitschritt der Länge $\Delta$ t\\
		t 		& technisch\\
		V 		& Vorlauf\\
		V 		& Verlust (Exergieanalyse)\\
		W		& Wärmeträgermedium\\
		
\end{longtable}
\end{onehalfspacing}
