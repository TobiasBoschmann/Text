\chapter{Methode}
\label{cha:Methode}

\section{Modelle der Wasserstoffkomponenten}

\subsection{Modellierung des Elektrolyseurs}
 Inhalt:\\
-für mein Modell gewählte Gleichungen\\
(Temp. und Druck Abhängigkeit der idealen Spannung, Überspannungen für die drei Technologien)\\ 
-Modellierung des Temperaturverhaltens\\
-Wichtig: Dynamik des Elektrolyseurs!!\\
(PEM nach Tjaarks keine Einschränkung, AEC dynamisch schlecht -> wie modellieren??, Idee: über Cp T/dt aber Cp Bestimmen für jeden Elektrolyseur nötig!\\ 
-Berücksichtigung weiterer Systemkomponenten im Modell


\paragraph{reversible Zellspannung}
Abhängigkeit der reversiblen Zellspannung von Druck und Temperatur.\\
\begin{align}
U_{rev} = U^0_{rev} + \Delta U_{rev}(p) + \Delta U_{rev}(T)
\end{align}
Nernst-Gleichung - Abhängigkeit der reversiblen Zellspannung $U_{rev}$ von den Partialdrücken der Produkte ($p_{H_2}$ für Wasserstoff und $p_{O_2}$ für Sauerstoff) sowie der Aktivität des Wassers ($a_{H_{2}O}$)\\
Festoxid Elektrolyse: gasförmiges Wasser-> Partialdruck des Wassers ($p_{H_{2}O}$) 
\begin{align}
 \Delta U_{rev}(p) = \frac{RT}{nF}\ln{(\frac{p_{H_2}\sqrt{p_{O_2}}}{a_{H_{2}O}})} \\ 
 \Delta U_{rev}(p) = \frac{RT}{nF}\ln{(\frac{p_{H_2}\sqrt{p_{O_2}}}{p_{H_{2}O}})}
\end{align}
Temperatur:\\
\begin{align}
	\Delta U_{rev}(T) = 8.5 \cdot 10^{-4} \cdot (T-298)
\end{align}
%???Gilt diese Gl. auch bei hohen Temperaturen (800 C bei SOEC)???\\

\paragraph{Aktivierungsverluste}
Für PEM und Alk:
\begin{align}
	U_{Akt} = \frac{T \cdot R}{z \cdot \alpha_k
		\cdot F} \cdot ln(i/i_0)\\ %PEM und Alk
\end{align}

\paragraph{Ohmsche Verluste}
Für PEM und Alk:\\

\begin{align}
	U_{Ohm} = \frac{\delta}{\sigma}\\ %PEM und Alk
	\sigma _{PEM} = (0.005139 \cdot \lambda -0.00326) \cdot exp(1268 \cdot (1/303-1/T))\\
	\sigma _{Alk} = -2.04m -0.0028m^2  + 0.005332mT +207.2 \frac{m}{T} +0.001043m^3-3 \cdot 10^{-3}m^2T^2
\end{align}
\textbf{R(T)} bestimmen?\\

\paragraph{Diffusionsüberspannung}
Für PEM und Alk:
\begin{align}
	U_{Diff} = - R \cdot T/(z \cdot F) \cdot log(1 - i/i_{max})
\end{align}

\paragraph{Modellierung der Thermodynamik}\ \\
Annahmen: stationär, T(Betrieb) = T(Austritt), i.G. u. i.Flüssigkeiten\\
Q>0 (Wärmeüberschuss) (weil bei i>0 -> U(rev)>U(tn) d.h. thermische Energie durch Verlustmech.)\\ 

Für PEM und Alk:\\
\begin{align}
	dU/dt = \dot{n} (h_{ein} - h_{aus}) + P_el - \dot{Q}\\
	0 = \dot{n}_{H2O, in} \cdot c_{p, H2O} \cdot (T_{H2O, in} - T) - I \cdot (U_{Zell} - U_{tn}) - \dot{Q}
\end{align}

Für SOEC, Wasser gasförmig -> i.G.\\


\subsection{Modellierung der Brennstoffzelle}
-für mein Modell gewählte Gleichungen\\

\section{Entwicklung von Energiesystem-Konzepten}
-Bewertungskriterien ausarbeiten\\
-Systemrandbedingungen festlegen\\
(H-Bedarf, Input-Temperaturen, Wärmebedarf, Stromkosten...)\\
-Konzepte entwickeln\\
-Festlegen von Parametern für Elektrolyseur, Brennstoffzelle usw. für die Simulation (Welche Leistungsklasse etc.)\\

